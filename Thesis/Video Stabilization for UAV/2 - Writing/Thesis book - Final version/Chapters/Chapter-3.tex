
\chapter{Problem Description}


\section{Overview}

In questa sezione si deve descrivere l'obiettivo della ricerca, le
problematiche affrontate ed eventuali definizioni preliminari nel
caso la tesi sia di carattere teorico.

\textbullet{} Describe Objetives

\textbullet{} Describe Problem


\section{Sigue el Vacile}

In mathematics a projective space is a set of elements similar to
the set P(V) of lines through the origin of a vector space V. The
cases when V=R2 or V=R3 are the projective line and the projective
plane, respectively. The idea of a projective space relates to perspective,
more precisely to the way an eye or a camera projects a 3D scene to
a 2D image. All points which lie on a projection line (i.e., a \textquotedbl{}line-of-sight\textquotedbl{}),
intersecting with the entrance pupil of the camera, are projected
onto a common image point. In this case the vector space is R3 with
the camera entrance pupil at the origin and the projective space corresponds
to the image points. Projective spaces can be studied as a separate
field in mathematics, but are also used in various applied fields,
geometry in particular. Geometric objects, such as points, lines,
or planes, can be given a representation as elements in projective
spaces based on homogeneous coordinates. As a result, various relations
between these objects can be described in a simpler way than is possible
without homogeneous coordinates. Furthermore, various statements in
geometry can be made more consistent and without exceptions. For example,
in the standard geometry for the plane two lines always intersect
at a point except when the lines are parallel. In a projective representation
of lines and points, however, such an intersection point exists even
for parallel lines, and it can be computed in the same way as other
intersection points. Other mathematical fields where projective spaces
play a significant role are topology, the theory of Lie groups and
algebraic groups, and their representation theories.


\subsection{Esta es una subsection}

In mathematics a projective space is a set of elements similar to
the set P(V) of lines through the origin of a vector space V. The
cases when V=R2 or V=R3 are the projective line and the projective
plane, respectively. The idea of a projective space relates to perspective,
more precisely to the way an eye or a camera projects a 3D scene to
a 2D image. All points which lie on a projection line (i.e., a \textquotedbl{}line-of-sight\textquotedbl{}),
intersecting with the entrance pupil of the camera, are projected
onto a common image point. In this case the vector space is R3 with
the camera entra


\subsubsection{Esto ya es el colmo de la subsub}

In mathematics a projective space is a set of elements similar to
the set P(V) of lines through the origin of a vector space V. The
cases when V=R2 or V=R3 are the projective line and the projective
plane, respectively. The idea of a projective space relates to perspective,
more precisely to the way an eye or a camera projects a 3D scene to
a 2D image. All points which lie on a projection line (i.e., a \textquotedbl{}line-of-sight\textquotedbl{}),
intersecting with the entrance pupil of the camera, are projected
onto a common image point. In this case the vector space is R3 with
the camera entrance pupil at the origin and the projective space corresponds
to the image points. Projective spaces can be studied as a separate
field in mathematics, but are also used in various applied fields,
geometry in particular. Geometric objects, such as points, lines,
or planes, can be given a representation as elements in projective
spaces based on homogeneous coordinates. As a result, various relations
between these objects can be described in a simpler way than is possible
without homogeneous coordinates. Furthermore, various statements in
geometry can be made more consistent and without exceptions. For example,
in the standard geometry for the plane two lines always intersect
at a point except when the lines are parallel. In a projective representation
of lines and points, however, such an intersection point exists even
for parallel lines, and it can be computed in the same way as other
intersection points. Other mathematical fields where projective spaces
play a significant role are topology, the theory of Lie groups and
algebraic groups, and their representation theories.

In mathematics a projective space is a set of elements similar to
the set P(V) of lines through the origin of a vector space V. The
cases when V=R2 or V=R3 are the projective line and the projective
plane, respectively. The idea of a projective space relates to perspective,
more precisely to the way an eye or a camera projects a 3D scene to
a 2D image. All points which lie on a projection line (i.e., a \textquotedbl{}line-of-sight\textquotedbl{}),
intersecting with the entrance pupil of the camera, are projected
onto a common image point. In this case the vector space is R3 with
the camera entrance pupil at the origin and the projective space corresponds
to the image points. Projective spaces can be studied as a separate
field in mathematics, but are also used in various applied fields,
geometry in particular. Geometric objects, such as points, lines,
or planes, can be given a representation as elements in projective
spaces based on homogeneous coordinates. As a result, various relations
between these objects can be described in a simpler way than is possible
without homogeneous coordinates. Furthermore, various statements in
geometry can be made more consistent and without exceptions. For example,
in the standard geometry for the plane two lines always intersect
at a point except when the lines are parallel. In a projective representation
of lines and points, however, such an intersection point exists even
for parallel lines, and it can be computed in the same way as other
intersection points. Other mathematical fields where projective spaces
play a significant role are topology, the theory of Lie groups and
algebraic groups, and their representation theories.
